%
%  Syllabus template for use with http://kjhealy.github.com/latex-custom-kjh
%
\documentclass[11pt,article,oneside]{memoir}

%% Script-based version control (requires vc package)
% \input{vc}     

\usepackage{graphicx, url}
\usepackage{rotating}        
\usepackage{memoir-article-styles} % in latex-custom-kjh/needs-memoir  
\usepackage{termcal}
\usepackage{tikz}
\DeclareGraphicsRule{.mps}{eps}{*}{}

% Define light grey color
\usepackage{xcolor}
\definecolor{lightgrey}{gray}{0.85}
\definecolor{uiblack}{RGB}{0, 0, 0}
\definecolor{uigold}{RGB}{255, 225, 0}
\definecolor{uigold2}{RGB}{246, 168, 0}

\usepackage[listings,skins]{tcolorbox}
\tcbset{colback=lightgrey, colframe = black, fonttitle=\large\bfseries, sharp corners,
 enhanced, attach boxed title to top left={xshift=-2mm,yshift=-2mm}, boxed title style={colback=black}}
%\newtcolorbox{mybox}[1]{colback=lightgrey,
%colframe=black,fonttitle=\bfseries,
%title=#1}

\tcbuselibrary{skins,breakable}
\newcounter{schedule}
\colorlet{colexam}{red!75!black}

\newtcolorbox[use counter=schedule]{schedule}[1]{%
empty,title={#1},attach boxed title to top left,
boxed title style={empty,size=minimal,toprule=2pt,top=4pt,
overlay={\draw[uiblack,line width=2pt]
([yshift=-1pt]frame.north west)--([yshift=-1pt]frame.north east);}},
coltitle=uiblack,fonttitle=\large\bfseries,
before=\par\medskip\noindent,parbox=false,boxsep=0pt,left=0pt,right=3mm,top=4pt,
breakable,pad at break=0mm,
overlay unbroken={\draw[uigold2,line width=1pt]
([yshift=-1pt]title.north east)--([xshift=-0.5pt,yshift=-1pt]title.north-|frame.east)
--([xshift=-0.5pt]frame.south east)--(frame.south west); },
overlay first={\draw[uigold2,line width=1pt]
([yshift=-1pt]title.north east)--([xshift=-0.5pt,yshift=-1pt]title.north-|frame.east)
--([xshift=-0.5pt]frame.south east); },
overlay middle={\draw[uigold2,line width=1pt] ([xshift=-0.5pt]frame.north east)
--([xshift=-0.5pt]frame.south east); },
overlay last={\draw[uigold2,line width=1pt] ([xshift=-0.5pt]frame.north east)
--([xshift=-0.5pt]frame.south east)--(frame.south west);},%
}


\newtcolorbox{resource}[1]{%
empty,title={#1},attach boxed title to top left,
boxed title style={empty,size=minimal,toprule=2pt,top=4pt,
overlay={\draw[uiblack,line width=2pt]
([yshift=-1pt]frame.north west)--([yshift=-1pt]frame.north east);}},
coltitle=uiblack,fonttitle=\Large\bfseries,
before=\par\medskip\noindent,parbox=false,boxsep=0pt,left=0pt,right=3mm,top=4pt,
breakable,pad at break=0mm,
overlay unbroken={\draw[uigold2,line width=1pt]
([yshift=-1pt]title.north east)--([xshift=-0.5pt,yshift=-1pt]title.north-|frame.east)
--([xshift=-0.5pt]frame.south east)--(frame.south west); },
overlay first={\draw[uigold2,line width=1pt]
([yshift=-1pt]title.north east)--([xshift=-0.5pt,yshift=-1pt]title.north-|frame.east)
--([xshift=-0.5pt]frame.south east); },
overlay middle={\draw[uigold2,line width=1pt] ([xshift=-0.5pt]frame.north east)
--([xshift=-0.5pt]frame.south east); },
overlay last={\draw[uigold2,line width=1pt] ([xshift=-0.5pt]frame.north east)
--([xshift=-0.5pt]frame.south east)--(frame.south west);},%
}


% increase margins
\usepackage[left=1.25in, right = .9in, top = .9in, bottom = 1in]{geometry}  
\usepackage{enumitem}

% making all description lists bold
\setlist[description]{font=\large\bfseries\itshape, style=nextline, noitemsep}

%% Choose font system. Comment out these lines if you are not using xelatex
\usepackage{fontspec}
\usepackage{xunicode} 

% Define light grey color
%\usepackage{xcolor}
%\definecolor{lightgrey}{gray}{0.9}

% Ajust quote environment to add more indentation on left side
%\newenvironment{myquote}{\list{}{\leftmargin=1.25in\rightmargin=0.5in}\mybox{lightgrey}{\item[]}{\endlist}}
\newenvironment{myquote}{\list{}{\leftmargin=1.25in\rightmargin=0.5in}\item[]}{\endlist}

% Biblatex
%\usepackage[american]{babel}
%\usepackage[babel]{csquotes}
%\usepackage[style=apa,
%            bibstyle=authoryear,
%            citestyle=authoryear-comp,
%            uniquename=false,
%            hyperref=true,
%            backend=biber, babel=hyphen, bibencoding=inputenc]{biblatex}

%% Fix biblatex's odd preference for using In: by default.
%\renewbibmacro{in:}{%
%  \ifentrytype{article}{}{%
%  \printtext{\bibstring{}\intitlepunct}}}


%% Links
%\usepackage[usenames,dvipsnames]{color}                     
\usepackage[xetex, 
	colorlinks=true,
	urlcolor=blue,
	plainpages=false,
  	pdfpagelabels, 
  	bookmarksnumbered
  	]{hyperref}   

\begin{document}
%%% xelatex font choices
%\defaultfontfeatures{}
%\defaultfontfeatures{Scale=MatchLowercase}    
% You will need to buy these fonts, change the names to fonts you own, or comment out if not using xelatex.      
\setromanfont[Mapping=tex-text]{Minion Pro} 
\setsansfont[Mapping=tex-text]{Myriad Pro} 
\setmonofont[Mapping=tex-text,Scale=0.8]{Pragmata} 

%% blank label items; hanging bibs for text
%% Custom hanging indent for vita items
\def\ind{\hangindent=1 true cm\hangafter=1 \noindent}
\def\labelitemi{$\cdot$}
%\renewcommand{\labelitemii}{~}

% Make figures as wide as the margins
\setkeys{Gin}{width=1\textwidth} 

%\chapterstyle{article-2}   % alternative styles are defined in latex-custom-kjh/needs-memoir/. Consider e.g.\chapterstyle{article-4}
%\pagestyle{kjh} 
  
%\title{\mytitle}     
%\author{\myauthor\smallskip\footnotesize\newline Office: 276 Sociology/Psychology \newline\texttt{\myemail}}
%\date{}

\begin{minipage}[b]{.7\linewidth}
\begin{flushleft}
{\huge PSQF 6250: Computer Packages for Statistical Analysis} \\[.1in]
{\large\sffamily Spring 2019 - Online Course} \\
\vspace*{.25in}
{\large Brandon LeBeau, Ph.D.} \\[.05in]
{\normalsize E-mail: \href{mailto:brandon-lebeau@uiowa.edu}{brandon-lebeau@uiowa.edu} \\Virtual Office Hours: T 12:30 pm to 2 pm or By appointment 
   \\ Department: Psychological and Quantitative Foundations, 361 LC \\ DEO: Dr. Ali, 361 LC}
\end{flushleft}
\end{minipage}
\begin{minipage}[t]{.3\linewidth}
\includegraphics[width=.9\linewidth]{DomeUIed}
\end{minipage}

\vspace{.25in}

\section*{Schedule}
% Week 1
\begin{schedule}{Week 1: January 14 - January 20}
\begin{description}
\item[Topics:] Reproducible Research; Markdown;  Installing R

\item[Readings:] R for Data Science -- \textit{Chapter 27}

\item[Homework:] Assignment 1 -- \textbf{Due January 31st}

\end{description}
\end{schedule}
% Week 2
\begin{schedule}{Week 2: January 21 - January 27}
\begin{description}
\item[Topics:] Graphics with R; R Basics

\item[Readings:] R for Data Science -- \textit{Chapters 1 -- 4}

\item[Homework:] 

\end{description}
\end{schedule}
% Week 3
\begin{schedule}{Week 3: January 28 - February 3}
\begin{description}
\item[Topics:] R Scripts; Data Munging

\item[Readings:] R for Data Science -- \textit{Chapters 5 -- 6}

\item[Homework:] Assignment 2 -- \textbf{Due February 14th}

\end{description}
\end{schedule}
% Week 4
\newpage
\begin{schedule}{Week 4: February 4 - February 10}
\begin{description}
\item[Topics:] Exploratory Data Analysis 

\item[Readings:] R for Data Science -- \textit{Chapter 7; 8 (opt)}

\item[Homework:] Assignment 3 -- \textbf{Due February 22nd}

\end{description}
\end{schedule}
% Week 5
\begin{schedule}{Week 5: February 11 - February 17}
\begin{description}
\item[Topics:] Data Import; Restructuring Data; Joining Data

\item[Readings:] R for Data Science -- \textit{Chapters 11 -- 13}

\item[Homework:] Assignment 4 -- \textbf{Due March 8th}
\end{description}
\end{schedule}
% Week 6
\begin{schedule}{Week 6: February 18 - February 24}
\begin{description}
\item[Topics:] Linear Model Building

\item[Readings:] R for Data Science -- \textit{Chapters 22 -- 24}

\item[Homework:]

\end{description}
\end{schedule}

% Week 7
\begin{schedule}{Week 7: February 25 - March 3}
\begin{description}
\item[Topics:] Model Assumptions

\item[Readings:] 

\item[Homework:] Assignment 5 -- \textbf{Due March 26th}

\end{description}
\end{schedule}

% Week 8
\begin{schedule}{Week 8: March 4 - March 10}
\begin{description}
\item[Topics:] Other Models 

\item[Readings:] 

\item[Homework:] Assignment 6 -- \textbf{Due April 10th}

\end{description}
\end{schedule}
% Week 9
\begin{schedule}{Week 9: March 11 - March 17}
\begin{description}
\item[Topics:] Reproducible Tables 

\item[Readings:] 

\item[Homework:] Project Proposal -- \textbf{Due April 4th (firm deadline)}

\end{description}
\end{schedule}

% Spring Break
\begin{schedule}{Spring Break: March 18 - March 24 -- No Class}

\end{schedule}


% Week 10
\begin{schedule}{Week 10: March 25 - March 31}
\begin{description}
\item[Topics:] Interactive Graphics

\item[Readings:] 

\item[Homework:] 

\end{description}
\end{schedule}
% Week 11
\begin{schedule}{Week 11: April 1 - April 7}
\begin{description}
\item[Topics:] Creating R Functions

\item[Readings:] 
\item[Homework:] Assignment 7 -- \textbf{Due April 20th}

\end{description}
\end{schedule}
% Week 12
\begin{schedule}{Week 12: April 8 - April 14}
\begin{description}
\item[Topics:] Creating R Functions

\item[Readings:] 
\item[Homework:] 

\end{description}
\end{schedule}
% Week 13
\begin{schedule}{Week 13: April 15 - April 21}
\begin{description}
\item[Topics:] Resampling and Bootstrapping

\item[Readings:] 

\item[Homework:]

\end{description}
\end{schedule}

% Week 14
\begin{schedule}{Week 14: April 22 - April 28}
\begin{description}
\item[Topics:] Monte Carlo Simulation with R

\item[Readings:] 

\item[Homework:] Assignment 8 -- \textbf{Due May 6th} \\
 All homework assignments \textbf{due May 6th}

\end{description}
\end{schedule}
% Week 15
\begin{schedule}{Week 15: April 29 - May 5}
\begin{description}
\item[Topics:] Version Control

\item[Readings:] git with github \\ git the simple guide

\item[Homework:] \textbf{Project due by May 9th at 11:59 pm}

\end{description}
\end{schedule}


\end{document}